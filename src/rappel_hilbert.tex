\section{Rappels sur les espaces de Hilbert}

\subsection{Définitions}

\begin{definition}
	Un espace de Hilbert $\Hilbert$ réel (resp complexe) est un espace vectoriel
	muni d'un produit scalaire (resp hermitien) qui est un espace de Banach
	pour la norme induite par le produit scalaire (resp hermitien).
\end{definition}

\begin{definition}
	Un produit scalaire hermitien est une fonction
	
	\begin{equation}
		\GSfunction{\dotprod{.}{.}}{\Hilbert \cartprod
		\Hilbert}{\complex} :
		(x, y) \rightarrow \dotprod{x}{y}
	\end{equation}

	vérifiant:

	\begin{enumerate}
		\item anti-linéaire en la première variable
		\item linéaire en la deuxième variable
		\item $\forall x, y \in \Hilbert, \, \dotprod{x}{y} =
			\conjuguate{\dotprod{y}{x}}$ (symétrie hermitienne)
		\item $\forall x \in \Hilbert, \, \dotprod{x}{x} \geq 0$
		\item $\dotprod{x}{x} = 0 \equiv x = 0_{\Hilbert}$
	\end{enumerate}
\end{definition}

\begin{theorem} [Représentation de Riesz]
	\label{thm:repr_riesz}
	On a $\Hilbert \GSid \GSdual{\Hilbert}$.
	L'identification est donné par l'application:

	\begin{equation}
		\GSfunction{i_{\Hilbert}}{\Hilbert}{\GSdual{\Hilbert}} : x \rightarrow
		i_{\Hilbert}(x) = x^{*}
	\end{equation}
	où $\GSfunction{x^{*}}{\Hilbert}{\mathbb{K}} : y \rightarrow
	\dotprod{x}{y}$
\end{theorem}

\subsection{Ajointe d'un opérateur}

Rappelons les notions d'application adjointe dans le cas d'espaces vectoriels
quelconques.

\begin{definition}
	\label{def:adjointe_dual_space}
	Soient $E$ et $F$ deux espaces vectoriels. Soient $\GSdual{E}$ et
	$\GSdual{F}$ leur espace dual respectif.
	Soit $T \in \GScontinueHomo{E}{F}$. On définit \textbf{l'adjointe de $T$},
	noté $T^{*}$:
	\begin{equation}
		\GSfunction{T}{\GSdual{F}}{\GSdual{E}} : f^{*} \rightarrow \adjointe{T}(f^{*})
	= f^{*} \circ T
		\label{def:adjointe_operator}
	\end{equation}
\end{definition}

Dans le cas de deux espaces de Hilbert $\Hilbert$ et $\mathcal{K}$, après
identification entre $\Hilbert$ et $\GSdual{\Hilbert}$ (resp. $\mathcal{K}$ et
$\GSdual{\mathcal{K}}$) on obtient une définition équivalente plus facile à
manipuler:

\begin{definition}
	\label{def:adjointe_hilbert_space}
	Soit $\Hilbert$ et $\mathcal{K}$ deux espaces de Hilbert.
	Soit $T \in \GScontinueHomo{\Hilbert}{\mathcal{K}}$.
	\textbf{L'adjointe de $T$}, notée $\adjointe{T}$, est l'unique application
	de $\GScontinueHomo{\mathcal{K}}{\Hilbert}$ tel que

	\begin{equation}
		\forall x \in \Hilbert, \, \forall y \in \mathcal{K}, \,
		{\dotprod{T(x)}{y}}_{\mathcal{K}} = {\dotprod{x}{\adjointe{T}(y)}}_{\Hilbert}
	\end{equation}
\end{definition}

Nous nous intéresserons particulièrement au cas $\Hilbert = \mathcal{K}$ (bien
que certaines propositions soient vraies quand on prend deux espaces de hilbert
différents.

\begin{proposition}
	Les deux définitions de l'adjointe coincident après identification.
\end{proposition}

On peut alors définir l'application:

\begin{equation}
	\GSfunction{\phi}{\GScontinueEndo{\Hilbert}}{\GScontinueEndo{\Hilbert}} : T \rightarrow
	\adjointe{T}
\end{equation}

Cette fonction $\phi$ possède quelques propriétés

\begin{proposition}
\begin{enumerate}
	\item $\forall S, T \in \GScontinueEndo{\Hilbert}$, $\phi(S + T) =
		\phi(S) + \phi(T)$
	\item $\forall S, T \in \GScontinueEndo{\Hilbert}$, $\phi(S \circ T) = \phi(T)
		\circ \phi(S)$
	\item $\forall S \in \GScontinueEndo{\Hilbert}$, $\forall \lambda \in
		\complex$, $\phi(\lambda S) = \conjuguate{\lambda} \, \phi(S)$.
	\item $\phi$ est involutive ($\phi \circ \phi =
		Id_{\GScontinueEndo{\Hilbert}}$)
\end{enumerate}
\end{proposition}

\begin{proof}
	On utilise la définition \ref{def:adjointe_hilbert_space}.
\end{proof}

On peut alors classer les opérateurs suivant leurs relations avec leur adjointe.

\begin{definition}
	Soit $T \in \GScontinueEndo{\Hilbert}$ et $\adjointe{T}$ son adjointe. On
	dit que
	\begin{itemize}
		\item $T$ est \textbf{normal} si $T \circ \adjointe{T} = \adjointe{T}
			\circ T$.
		\item $T$ est \textbf{unitaire} si $T^{-1} = \adjointe{T}$.
		\item $T$ est \textbf{autoadjointe} ou \textbf{hermitique} si $T =
			\adjointe{T}$.
	\end{itemize}
\end{definition}

\begin{remarque}
	Un opérateur unitaire est en particulier normal. Et pour un opérateur
	unitaire, on a $T \circ \adjointe{T} = \adjointe{T} \circ T =
	Id_{\Hilbert}$.
\end{remarque}

\begin{exercice}
	Montrer qu'on a, pour tout sous-espace $G$ de $\Hilbert$:
	$T(G) \subseteq G \equiv T^{*}(\GSortho{G}) \subseteq \GSortho{G}$.
\end{exercice}

\begin{exercice}
	Soit $T$ un opérateur normal. Alors $\ker(T) = \ker(\adjointe{T})$.
\end{exercice}

\begin{exercice}
	\label{ex:ker_image_ortho_adjointe}
	Soit $T \in \GScontinueEndo{\Hilbert}$, et $\adjointe{T}$.

	Alors $\ker{\adjointe{T}} = \GSortho{(Im(T))}$.
\end{exercice}

\subsection{Base hilbertienne}

Rappelons également quelques notions vues l'année passée.

\begin{definition}
	Soit $\Hilbert$ un espace de Hilbert.
	Soit $\GSsequence{e}{n}{\integer} \subseteq \Hilbert$.

	On dit que $\GSsequence{e}{n}{\integer}$ est \textbf{une famille
	orthonormée} s'ils sont orthogonaux 2 à 2 et tous de norme $1$.
\end{definition}

\begin{definition}
	Soit $\Hilbert$ un espace de Hilbert.
	Soit $\GSsequence{e}{n}{\integer} \subseteq \Hilbert$ une famille
	orthonormée.

	On dit que $\GSsequence{e}{n}{\integer}$ est \textbf{une base de Hilbert} si

	\begin{equation}
	\forall x \in \Hilbert, \, x = \displaystyle \sum_{n = -\infty}^{\infty}
	\dotprod{x}{e_{n}} e_{n}
	\end{equation}
\end{definition}

% Tout espace de Hilbert admet une base hilbertienne.

