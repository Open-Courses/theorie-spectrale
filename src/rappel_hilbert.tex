\section{Rappels sur les espaces de Hilbert}

\begin{definition}
	Un espace de Hilbert $\Hilbert$ réel (resp complexe) est un espace vectoriel
	muni d'un produit scalaire (resp hermitienne) qui est un espace de Banach
	pour la norme induite par le produit scalaire (resp hermitienne).
\end{definition}

\begin{theorem}
	$\GSdual{\Hilbert} \GSid \Hilbert$ (identification)
	\label{thm:repr_riesz}
\end{theorem}

Rappelons les notions d'application adjointe.

\begin{definition}
	Soient $E$ et $F$ deux espaces vectoriels. Soient $\GSdual{E}$ et
	$\GSdual{F}$ leur espace dual respectif.
	Soit $T \in \GScontinueHomo{E}{F}$. On définit l'adjointe de $T$, noté
	$T^{*}$:
	\begin{equation}
		\GSfunction{T}{\GSdual{F}}{\GSdual{E}} : f^{*} \rightarrow T^{*}(f^{*})
	= f^{*} \circ T
		\label{def:adjointe_operator}
	\end{equation}
\end{definition}

On a vu en algèbre linéaire que l'application adjointe est l'unique application
tel que $\forall x, y \in E$ $\dotprod{x}{T(y)} = \dotprod{T^{*}(x)}{y}$.

Dans le cas des espaces de Hilbert, après l'identification entre
$\GSdual{\Hilbert}$ et $\Hilbert$, la définition de l'adjointe vue en algèbre
linéaire est une propriété.

Nous avons donc $\adjointe{T} \in \GScontinueEndo{\Hilbert}$.

On peut alors définir la fonction 

\begin{equation}
	\GSfunction{\phi}{\GScontinueEndo{\Hilbert}}{\GScontinueEndo{\Hilbert}} : T \rightarrow
	\adjointe{T}
\end{equation}

Cette fonction $\phi$ possède quelques propriétés

\begin{enumerate}
	\item $\forall S, T \in \GScontinueEndo{\Hilbert}$, $\phi(S + T) =
		\phi(S) + \phi(T)$
	\item $\forall S, T \in \GScontinueEndo{\Hilbert}$, $\phi(S \circ T) = \phi(T)
		\circ \phi(S)$
	\item $\forall S \in \GScontinueEndo{\Hilbert}$, $\forall \lambda \in
		\complex$, $\phi(\lambda S) = \conjuguate{\lambda} \, \phi(S)$.
\end{enumerate}

Rappelons également quelques notions vues l'année passée.

\begin{definition}
	Soit $\Hilbert$ un espace de Hilbert.
	Soit $\GSsequence{e}{n}{\integer} \subseteq \Hilbert$.

	On dit que $\GSsequence{e}{n}{\integer}$ est \textbf{une famille
	orthonormée} s'ils sont orthogonaux 2 à 2 et tous de norme $1$.
\end{definition}

\begin{definition}
	Soit $\Hilbert$ un espace de Hilbert.
	Soit $\GSsequence{e}{n}{\integer} \subseteq \Hilbert$ une famille
	orthonormée.

	On dit que $\GSsequence{e}{n}{\integer}$ est \textbf{une base de Hilbert} si

	\begin{equation}
	\forall x \in \Hilbert, \, x = \displaystyle \sum_{n = -\infty}^{\infty}
	\dotprod{x}{e_{n}} e_{n}
	\end{equation}

\end{definition}

\begin{proposition}
	
\end{proposition}
