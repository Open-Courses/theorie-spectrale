\section{Rappels sur les espaces de Hilbert}

\begin{definition}
	Un espace de Hilbert $\Hilbert$ réel (resp complexe) est un espace vectoriel
	muni d'un produit scalaire (resp hermitienne) qui est un espace de Banach
	pour la norme induite par le produit scalaire (resp hermitienne).
\end{definition}

\begin{theorem}
	$\GSdual{\Hilbert} \GSid \Hilbert$ (identification)
	\label{thm:repr_riesz}
\end{theorem}

Rappelons les notions d'application adjointe.

\begin{definition}
	Soient $E$ et $F$ deux espaces vectoriels. Soient $\GSdual{E}$ et
	$\GSdual{F}$ leur espace dual respectif.
	Soit $T \in \GScontinueHomo{E}{F}$. On définit l'adjointe de $T$, noté
	$T^{*}$:
	\begin{equation}
		\GSfunction{T}{\GSdual{F}}{\GSdual{E}} : f^{*} \rightarrow T^{*}(f^{*})
		= f \circ T
		\label{def:adjointe_operator}
	\end{equation}
\end{definition}

On a vu en algèbre linéaire que l'application adjointe est l'unique application
tel que $\forall x, y \in E$ $\dotprod{x}{T(y)} = \dotprod{T^{*}(x)}{y}$.

Dans le cas des espaces de Hilbert, après l'identification entre
$\GSdual{\Hilbert}$ et $\Hilbert$, la définition de l'adjointe vue en algèbre
linéaire est une propriété.

