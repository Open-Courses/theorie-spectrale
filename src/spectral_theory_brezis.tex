\section{Théorie spectrale: Brezis}

Le bus de cette section va être de retrouver certaines propriétés en dimension
quelconque que nous avons en dimension finie pour les opérateurs linéaires, et
qui nous ont permis de construire la théorie spectrale en dimension finie.

%Pour la première propriété, nous voulons que si $T$ est bijective, alors elle
%est un isomorphisme de $E$. Cette propriété est acceptée si $E$ est un espace de
%Banach. Nous allons donc poser comme première condition que $E$ soit un espace
%de Banach.

%Cette demande n'est pas restrictive quand on est en dimension finie. En effet,
%comme nous travaillons sur $\complex$ ou $\real$, chaque espace de dimension
%finie est de Banach.

%Un type d'opérateur, appelé \textbf{opérateur compacte}, répond à ces critères.

% Rappel de relativement compact

\subsection{Opérateur compact}

\begin{definition}
	Soit $T \in \GScontinueHomo{E}{F}$ où $E$ et $F$ sont deux espaces
	vectoriels normés.

	Alors $T$ est dit \textbf{compacte} si $T(B_{E})$ est relativement compacte
	dans $F$.
\end{definition}

De manière équivalente, on peut définir les opérateurs compacts comme les
opérateurs envoyant toutes parties bornées dans une partie relativement
compacte.

On note $\GScompactHomo{E}{F}$ l'ensemble des opérateurs compactes.

\begin{proposition}
	$\GScompactHomo{E}{F}$ est un sous-espace vectoriel fermé de
	$\GScontinueHomo{E}{F}$.
\end{proposition}

\begin{proof}
	
\end{proof}

\begin{proposition}
	Soit $T \in \GScompactEndo{E}$. Alors:

\begin{enumerate}
	\item $\ker(Id_{E} - T)$ est de dimension finie.
	\item $(Id_{E} - T)$ est injective $\equiv$ $(Id_{E} - T)$ est surjective
		$\equiv$ $(Id_{E} - T) \in Isom(E)$.
	\item $Im(Id_{E} - T)$ est fermé et $\GSortho{\ker(Id_{E} - \adjointe{T})} = Im(Id_{E}
		- T)$.
\end{enumerate}
\end{proposition}

\begin{proof}
	
\end{proof}

\begin{definition}
	Soit $T \in \GScompactEndo{E}$. On dit que $T$ est \textbf{de rang fini}
	si son image $Im(T)$ est de dimension finie.
\end{definition}

\begin{proposition}
	
\end{proposition}

\subsection{Spectre d'un oéprateur compact}

\begin{definition}
	Soit $T \in \GScontinueEndo{E}$.

	On définit \textbf{le résolvant de $T$} par l'ensemble:

	\begin{equation}
		\resolvant{T} = \GSset{\lambda \in \complex}{T - \lambda
			{Id}_{E} \in \GSisomorphism{E}}
			\label{def:resolvant_operator}
	\end{equation}

	On définit le complémentaire de $\rho(T)$ comme \textbf{le spectre de $T$}.
	On a donc que le spectre de $T$:

	\begin{equation}
		\spectrum{T} = \GSset{\lambda \in \complex}{T - \lambda {Id}_{E}
		\notin \GSisomorphism{E}}
		\label{def:spectrum_operator}
	\end{equation}
\end{definition}

Le spectre est donc un sous-ensemble de $\complex$.

Tout d'abord, montrons que le spectre et l'ensemble des valeurs propres sont
deux notions différentes quand on travaille en dimension infinie. Cependant,
comme nous l'avons vu, le spectre et les valeurs propres forment le même
ensemble en dimension finie vu le théorème du rang.

\begin{proposition}
	Soit $T \in \GScontinueEndo{E}$.
	Soit $\lambda \in \complex$ valeur propre de $T$.

	Alors $\lambda \in \spectrum{T}$.
\end{proposition}

\begin{proof}
	Si $\lambda$ est valeur propre, alors $T - \lambda Id_{E}$ n'est pas
	inversible car il existe $v \in E$ tel que $v \neq 0_{E}$ et $T(v) - \lambda
	v = 0$. Donc $v \in \ker{T - \lambda Id_{E}}$, et donc $T - \lambda Id_{E}$
	n'est pas injective.
\end{proof}

Nous avons donc que le spectre de $T$ contient toutes les valeurs propres de
$T$.
Nous montrerons avec un exemple que nous avons déjà rencontré que le spectre ne
contient pas uniquement les valeurs propres.

Les isométries de $E$ sont étroitement liées à la norme définie sur $E$. On peut
donc penser que le spectre diffère d'une norme à l'autre, ou du moins qu'elle a
un lien avec la topologie de la norme.

Le spectre est en effet liée à la topologie, et possède une propriété commune à
chaque norme.

\begin{proposition}
	Soit $T \in \GScontinueEndo{E}$.
	Le spectre de $T$ est un compact de $\complex$.
\end{proposition}

Avant de montrer cette proposition, nous allons démontrer quelques lemmes.
