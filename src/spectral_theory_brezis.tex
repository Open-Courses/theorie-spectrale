\section{Théorie spectrale : Brezis}

Le bus de cette section va être de retrouver certaines propriétés que nous avons
en dimension finie pour les opérateurs linéaires, et qui nous ont permis de
construire la théorie spectrale en dimension finie, mais en dimension
quelconque.

Prenons un opérateur $T \in \GScontinueEndo{E}$ où $E$ est un espace vectoriel
normé.  Nous allons chercher des opérateurs qui vont pouvoir répondre aux
propriétés

\begin{enumerate}
	\item $T$ est injective $\equiv$ $T$ est surjective $\equiv$ $T \in
		Isom(E)$.
	\item $\ker(T)$ est de dimension finie.
	\item $Im(T)$ est fermé et 
\end{enumerate}

Pour la première propriété, nous voulons que si $T$ est bijective, alors elle
est un isomorphisme de $E$. Cette propriété est acceptée si $E$ est un espace de
Banach. Nous allons donc poser comme première condition que $E$ soit un espace
de Banach.

Cette demande n'est pas restrictive quand on est en dimension finie. En effet,
comme nous travaillons sur $\complex$ ou $\real$, chaque espace de dimension
finie est de Banach.
