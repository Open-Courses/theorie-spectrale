\section{Théorie spectrale : Brezis}

Le bus de cette section va être de retrouver certaines propriétés que nous avons
en dimension finie pour les opérateurs linéaires, et qui nous ont permis de
construire la théorie spectrale en dimension finie, mais en dimension
quelconque.

Prenons un opérateur $T \in \GScontinueEndo{E}$ où $E$ est un espace vectoriel
normé.  Nous allons chercher des opérateurs qui vont pouvoir répondre aux
propriétés

\begin{enumerate}
	\item $T$ est injective $\equiv$ $T$ est surjective $\equiv$ $T \in
		Isom(E)$.
	\item $\ker(T)$ est de dimension finie.
	\item $Im(T)$ est fermé et 
\end{enumerate}

Pour la première propriété, nous voulons que si $T$ est bijective, alors elle
est un isomorphisme de $E$. Cette propriété est acceptée si $E$ est un espace de
Banach. Nous allons donc poser comme première condition que $E$ soit un espace
de Banach.

Cette demande n'est pas restrictive quand on est en dimension finie. En effet,
comme nous travaillons sur $\complex$ ou $\real$, chaque espace de dimension
finie est de Banach.

\begin{definition}
	Soit $T \in \GScontinueEndo{E}$.

	On définit \textbf{le résolvant de $T$} par l'ensemble:

	\begin{equation}
		\resolvant{T} = \GSset{\lambda \in \complex}{T - \lambda
			{Id}_{E} \in \GSisometryEndo{E}}
			\label{def:resolvant_operator}
	\end{equation}

	On définit le complémentaire de $\rho(T)$ comme \textbf{le spectre de $T$}.
	On a donc que le spectre de $T$:

	\begin{equation}
		\spectrum{T} = \GSset{\lambda \in \complex}{T - \lambda {Id}_{E}
		\notin \GSisometryEndo{E}}
		\label{def:spectrum_operator}
	\end{equation}
\end{definition}

Le spectre est donc un sous-ensemble de $\complex$.

Tout d'abord, montrons que le spectre et l'ensemble des valeurs propres sont
deux notions différentes quand on travaille en dimension infinie. Cependant,
comme nous l'avons vu, le spectre et les valeurs propres forment le même
ensemble en dimension finie vu le théorème du rang.

\begin{proposition}
	Soit $T \in \GScontinueEndo{E}$.
	Soit $\lambda \in \complex$ valeur propre de $T$.

	Alors $\lambda \in \spectrum{T}$.
\end{proposition}

\begin{proof}
	Si $\lambda$ est valeur propre, alors $T - \lambda Id_{E}$ n'est pas
	inversible car il existe $v \in E$ tel que $v \neq 0_{E}$ et $T(v) - \lambda
	v = 0$. Donc $v \in \ker{T - \lambda Id_{E}}$, et donc $T - \lambda Id_{E}$
	n'est pas injective.
\end{proof}

Nous avons donc que le spectre de $T$ contient toutes les valeurs propres de
$T$.
Nous montrerons avec un exemple que nous avons déjà rencontré que le spectre ne
contient pas uniquement les valeurs propres.

Les isométries de $E$ sont étroitement liées à la norme définie sur $E$. On peut
donc penser que le spectre diffère d'une norme à l'autre, ou du moins qu'elle a
un lien avec la topologie de la norme.

Le spectre est en effet liée à la topologie, et possède une propriété commune à
chaque norme.

Nous allons d'abord montrer un petit lemme 
\begin{proposition}
	Soit $T \in \GScontinueEndo{E}$.
	Le spectre de $T$ est un compact de $\complex$.
\end{proposition}


