\section{Introduction}


%Ce cours sur la théorie spectrale est basé sur la structure du Bresiz ainsi que
%d'un PDF réalisé par Emmanuel Fricain.
%Le premier aborde la théorie spectrale sur les espaces de Banach et les
%espaces de Hilbert \textbf{réel}, tandis que le deuxième prend en généralité sur
%les espaces de Hilbert complexes et réels sans parler d'espace de Banach.

%Cependant, Emmanuel Fricain développe l'idée de la théorie spectrale plus
%profondément en étudiant également les algèbres de Banach et en donnant quelques
%proposition sur les $C^{*}$-algèbres.

%J'ai tenté de rassembler les deux ouvrages pour pouvoir tirer les mêmes
%conclusions qu'eux sur les espaces de Banach et les espaces
%de Hilbert quelconques quand cela est possible. Lorsque nous sommes dans des cas
%spécifiques au corps réel ou complex, il sera spécifié.

-- Plan du cours en donnant les références. On part de l'algèbre linéaire II,
pour arriver aux algèbres. Voir plan papier. Expliquer qu'en dimension finie, ce
qu'on va voir reste toujours vrai parce qu'en fait tous (à préciser) les
opérateurs sont les mêmes.

-- Expliquer l'importance de la théorie spectrale pour les opérateurs. Pour
cela, je divise le tableau en deux, et le plan du cours sera mis sur un côté
avec les références, tandis que l'utilité sera mis en face sur l'autre partie du
tableau.
