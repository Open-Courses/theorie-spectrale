\section{Introduction}


%Ce cours sur la théorie spectrale est basé sur la structure du Bresiz ainsi que
%d'un PDF réalisé par Emmanuel Fricain.
%Le premier aborde la théorie spectrale sur les espaces de Banach et les
%espaces de Hilbert \textbf{réel}, tandis que le deuxième prend en généralité sur
%les espaces de Hilbert complexes et réels sans parler d'espace de Banach.

%Cependant, Emmanuel Fricain développe l'idée de la théorie spectrale plus
%profondément en étudiant également les algèbres de Banach et en donnant quelques
%proposition sur les $C^{*}$-algèbres.

%J'ai tenté de rassembler les deux ouvrages pour pouvoir tirer les mêmes
%conclusions qu'eux sur les espaces de Banach et les espaces
%de Hilbert quelconques quand cela est possible. Lorsque nous sommes dans des cas
%spécifiques au corps réel ou complex, il sera spécifié.

-- Plan du cours en donnant les références. On part de l'algèbre linéaire II,
pour arriver aux algèbres. Voir plan papier. Expliquer qu'en dimension finie, ce
qu'on va voir reste toujours vrai parce qu'en fait tous (à préciser) les
opérateurs sont les mêmes.

-- Expliquer l'importance de la théorie spectrale pour les opérateurs. Pour
cela, je divise le tableau en deux, et le plan du cours sera mis sur un côté
avec les références, tandis que l'utilité sera mis en face sur l'autre partie du
tableau.

D'abord expliquer la méthode pratique de la théorie spectrale, c'est-à-dire
diviser l'espace tout entier en sous-espace sur lesquels la fonction est
'sympathique'. Cela aura une utilité en mécanique quantique.

(faire un schéma qui montre qu'à chaque mesure [réelle, physique, on peut
'voir'] est associé une observable)

-- En mécanique quantique, on étudie les opérateurs linéaires continues sur un
espace de hilbert $\Hilbert$ complexe, appelé \textbf{l'espace des états}
($\Hilbert$ est souvent un espace d'opérateurs, comme
$\mathcal{L}^{2}(\real^{n}, \complex)$) et
particulièrement les valeurs propres et les sous-espaces propres associés.
A chaque mesure que nous pouvons faire, on associe un opérateur de
$\GScontinueEndo{\Hilbert}$.
Les valeurs que nous pouvons observer sont les valeurs propres (qui
correspondent au spectre sans l'origine sous certaines conditions), donc l'étude
du spectre joue un rôle essentiel, et l'état du système est déterminé
\textbf{ENTIÈREMENT} grace un élément de l'espace de Hilbert $\Hilbert$.

Cependant, en mécanique quantique, nous supposons que les opérateurs associés à
une mesure répondent à deux critères: hermitiques et décomposables en une base
de vecteur propre.

-- Très peu de preuves, par manque de temps, vont être démontrées. Celles qui le
seront donneront les résultats essentiels sur lesquels je veux insister. Ce
cours est là surtout pour apporter des notions diverses, comme lors d'un
séminaire.

