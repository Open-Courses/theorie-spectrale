\section{Rappels d'algèbre linéaire}

Soit $E$ un espace vectoriel de \textit{dimension finie} sur un corps
$\mathbb{K}$.

\begin{proposition}
	$\GSendomorphism{E} = \GScontinueEndo{E}$
	(Chaque application linéaire est continue)
\end{proposition}

\begin{proof}
	Analyse III Crespo
\end{proof}


On gardera la notation $\GSendomorphism{E}$ pour être cohérent avec le cours
d'algèbre linéaire II.

\begin{definition}
	\label{def:alg_lin_valeur_propre}
	Soit $f \in \GSendomorphism{E}$.
	Soit $\lambda \in \mathbb{K}$.
	On dit que $\lambda$ est \textbf{valeur propre} de $f$ si $\ker(f - \lambda
		Id_{E}) \neq \left\{ 0_{E} \right\}$.
	
	On appelle $\ker(f - \lambda Id_{E})$ \textbf{le sous-espace propre associé
	à (la valeur propre) $\lambda$}
\end{definition}

\begin{definition}
	\label{def:alg_lin_vecteur_propre}
	Soit $f \in \GSendomorphism{E}$
	Soit $v \in E$.
	On dit que $v$ est \textbf{vecteur propre} de $E$ s'il est contenu dans un
	sous-espace propre.
\end{definition}

\begin{proposition}
	Soit $f \in \GSendomorphism{E}$.
	Alors les assertions suivantes sont équivalentes:

	\begin{enumerate}
		\item $f$ est injective
		\item $f$ est surjective
		\item $f \in \GSisomorphism{E}$.
	\end{enumerate}
\end{proposition}

\begin{proof}
	Application du théorème du rang.
\end{proof}
