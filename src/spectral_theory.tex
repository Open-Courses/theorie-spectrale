\section{Théorie spectrale}

Le bus de cette section va être de retrouver certaines propriétés en dimension
quelconque que nous avons en dimension finie pour les opérateurs linéaires, et
qui nous ont permis de construire la théorie spectrale en dimension finie.

Nous allons étudier une classe spécifique d'opérateur linéaire: les opérateurs
compacts.

Ceux-ci ont des caractéristiques intéressantes: l'injectivité et le surjectivité
ne sont plus deux choses disjointes (comme en dimension finie), leurs valeurs
propres et leur spectre coincident en dehors de l'origine (ce qui n'est pas
toujours la cas en dimension infinie), et forment un sous-espace fermé de tous
les opérateurs.

Nous allons principalement travailler avec des espaces de Banach pour éviter de
se soucier des opérateurs bijective, linéaire et continues qui ne sont pas des
isomorphismes.

%Pour la première propriété, nous voulons que si $T$ est bijective, alors elle
%est un isomorphisme de $E$. Cette propriété est acceptée si $E$ est un espace de
%Banach. Nous allons donc poser comme première condition que $E$ soit un espace
%de Banach.

%Cette demande n'est pas restrictive quand on est en dimension finie. En effet,
%comme nous travaillons sur $\complex$ ou $\real$, chaque espace de dimension
%finie est de Banach.

%Un type d'opérateur, appelé \textbf{opérateur compacte}, répond à ces critères.

Rappelons qu'on peut définir un ensemble \textbf{relativement compact} par la
définition d'être contenue dans un compact, ou de manière équivalente, que son
adhérence est compacte.

\subsection{Opérateur compact dans les espaces de Banach réels ou complexes}

\begin{definition}
	Soit $T \in \GScontinueHomo{E}{F}$ où $E$ et $F$ sont deux espaces
	vectoriels normés.

	Alors $T$ est dit \textbf{compacte} si $T(\GSunitBoule{E})$ est relativement compacte
	dans $F$.
\end{definition}

De manière équivalente, on peut définir les opérateurs compacts comme les
opérateurs envoyant toutes parties bornées dans une partie relativement
compacte (exercice).

En particulier, on peut utiliser la définition avec la boule unité ouverte peut
être remplacée par la boule unité fermée.

On note $\GScompactHomo{E}{F}$ l'ensemble des opérateurs compactes.

\begin{proposition}
	$\GScompactHomo{E}{F}$ est un sous-espace vectoriel fermé de
	$\GScontinueHomo{E}{F}$.

	C'est-à-dire que la propriété d'être compacte est 'stable par passage à la
	limite'.
\end{proposition}

\begin{proof}
	-- Montrons d'abord que c'est un sous-espace vectoriel.

	Si on prend deux opérateur compacts $S$ et $T$, ainsi que $\lambda \in
	\complex$, nous devons montrer que $S + T$ est compact ainsi que $\lambda S$
	est compact.

	On a $\adh{(S + T)(\GSunitBoule{E})} =  \adh{S(\GSunitBoule{E}) +
	T(\GSunitBoule{E})} \subseteq \adh{S(\GSunitBoule{E})} +
	\adh{T(\GSunitBoule{E})}$. La compacité étant stable par somme, on a bien
	que $(S + T)$ est compact car $(S + T)(\GSunitBoule{E})$ est relativement
	compacte.

	On a $(\lambda S)(\GSunitBoule{E}) = S(\lambda \GSunitBoule{E})$ par
	linéarité de $S$. Comme $S$ est compact et $\lambda \GSunitBoule{E}$ est
	bornée, alors $S(\lambda \GSunitBoule{E})$ est relativement compact. Donc
	$\lambda S$ est compact.

	-- Montrons que $\GScompactHomo{E}{F}$ est fermé.
\end{proof}

La propriété d'être un opérateur compact est aussi stable par composition.

\begin{proposition}
	Soient $E, F, G$ des espaces de Banach.
	Soient $S \in \GScompactHomo{E}{F}$ et $T \in \GScompactHomo{F}{G}$
	Alors $T \circ S \in \GScompactHomo{E}{G}$.
\end{proposition}

\begin{proof}
	On a $S(\GSunitBoule{E})$ qui est compris dans un compact $K$ de $F$. Comme $T$ est
	continue, $T(K)$ est aussi un compact de $G$. Donc $(T \circ
	S)(\GSunitBoule{E}) \subseteq T(K)$, d'où $T \circ S$ est compact.
\end{proof}

\begin{definition}
	Soit $T \in \GScontinueHomo{E}{F}$. On dit que $T$ est \textbf{de rang fini}
	si son image $Im(T) = T(E)$ est de dimension finie.

	On note $\GSfiniteRankHomo{E}{F}$ l'ensemble des opérateurs de rang fini.
\end{definition}

% ----- Opérateurs de rang fini
\begin{proposition}
	$\GSfiniteRankHomo{E}{F}$ est un sous-espace vectoriel de
	$\GScontinueHomo{E}{F}$.
\end{proposition}

\begin{proof}
	Si on prend deux opérateur $S$ et $T$ dont l'image est de dimension finie,
	l'image de $S + T$ est de dimension inférieur ou égale à $dim(Im(S)) +
	dim(Im(T))$ qui est finie.

	Pour $\lambda \in \mathbb{K}$ et $S$ de rang fini, on a $\lambda S(E) =
	S(\lambda E) = S(E)$ qui est de dimension finie.
\end{proof}

\begin{proposition}
	On a $\GSfiniteRankHomo{E}{F} \subseteq \GScompactHomo{E}{F}$. C'est-à-dire
	que tout opérateur linéaire continue de $E$ dans $F$ avec une image de
	dimension finies est compact.
\end{proposition}

\begin{proof}
	
\end{proof}

% ----- Opérateurs de Fredholm

\begin{proposition}
	Soit $T \in \GScompactEndo{E}$. Alors:

\begin{enumerate}
	\item $\ker(Id_{E} - T)$ est de dimension finie.
	\item $(Id_{E} - T)$ est injective $\equiv$ $(Id_{E} - T)$ est surjective
		$\equiv$ $(Id_{E} - T) \in Isom(E)$.
	\item $Im(Id_{E} - T)$ est fermé et $\GSortho{\ker(Id_{E} - \adjointe{T})} = Im(Id_{E}
		- T)$.
\end{enumerate}
\end{proposition}

\begin{proof}
	
\end{proof}

% ----- Proposition sur l'adjointe de T où T est compact.
%\begin{proposition}
	%Soit $T \in \GScontinueEndo{\Hilbert}$.

	%Alors $T \in \GScompactEndo{\Hilbert} \equiv \adjointe{T} \in
	%\GScompactEndo{\GSortho{\Hilbert}}$.
%\end{proposition}

%\begin{proof}
	
%\end{proof}

\subsection{Spectre d'un oéprateur compact}

Nous travaillerons par la suite sur le même espace de départ et d'arrivée. $E$
sera un espace de Banach, et $\Hilbert$ un espace de Hilbert.
Rappelons qu'un espace de Hilbert est un espace de Banach en particulier, donc
les propriétés vraies dans les espaces de Banach sont vraies dans les espaces de
Hilbert.

\begin{definition}
	Soit $T \in \GScontinueEndo{E}$.

	On définit \textbf{le résolvant de $T$} par l'ensemble:

	\begin{equation}
		\resolvant{T} = \GSset{\lambda \in \complex}{T - \lambda
			{Id}_{E} \in \GSisomorphism{E}}
			\label{def:resolvant_operator}
	\end{equation}

	On pose alors pour chaque $\lambda \in \resolvant{T}$, $R_{\lambda}(T) = (T -
	\lambda Id_{E})^{-1}$.

	On définit le complémentaire de $\rho(T)$ comme \textbf{le spectre de $T$},
	et on le note $\spectrum{T}$.
	On a donc que le spectre de $T$:

	\begin{equation}
		\spectrum{T} = \GSset{\lambda \in \complex}{T - \lambda {Id}_{E}
		\notin \GSisomorphism{E}}
		\label{def:spectrum_operator}
	\end{equation}
\end{definition}

Le spectre est donc un sous-ensemble de $\complex$.

L'exemple du shift nous montre que les valeurs propres ne couvrent pas tout le
spectre.
Cependant, comme nous l'avons vu, le spectre et les valeurs propres forment le
même ensemble en dimension finie vu le théorème du rang.

\begin{proposition}
	Soit $T \in \GScontinueEndo{E}$.
	Soit $\lambda \in \complex$ valeur propre de $T$.

	Alors $\lambda \in \spectrum{T}$.
\end{proposition}

\begin{proof}
	Si $\lambda$ est valeur propre, alors $T - \lambda Id_{E}$ n'est pas
	inversible car il existe $v \in E$ tel que $v \neq 0_{E}$ et $T(v) - \lambda
	v = 0$. Donc $v \in \ker{T - \lambda Id_{E}}$, et donc $T - \lambda Id_{E}$
	n'est pas injective.
\end{proof}

Nous avons donc que le spectre de $T$ contient toutes les valeurs propres de
$T$.

Les isométries de $E$ sont étroitement liées à la norme définie sur $E$. On peut
donc penser que le spectre diffère d'une norme à l'autre, ou du moins qu'elle a
un lien avec la topologie de la norme définie sur $E$.

Le spectre est en effet liée à la topologie, et possède une propriété commune à
chaque norme.

\begin{theorem}
	\label{thm:spectrum_compact}
	Soit $T \in \GScontinueEndo{E}$.
	Le spectre de $T$ est un compact de $\complex$.
\end{theorem}

Avant de montrer cette théorème, nous allons démontrer quelques lemmes, et
donner quelques rappels.

\begin{lemma}
	$T - \lambda Id_{E}$ et $T - \lambda_{0} Id_{E}$ commutent.
\end{lemma}

\begin{proof}
	
\end{proof}

\begin{proposition} [Identité de la résolvante]
	\label{prop:resolvante_identity}
	Soit $T \in \GScompactEndo{E}$.
	On a pour tout $\lambda, \mu \in \resolvant{T}$:

	\begin{equation}
		R_{\lambda}(T) - R_{\mu}(T) = (\lambda - \mu) R_{\lambda}(T) \circ
		R_{\mu}(T)
	\end{equation}
\end{proposition}

\begin{proof}

\end{proof}

\begin{proposition}
	-- \GSisomorphismHomo{E}{F} est un ouvert de $\GScontinueHomo{E}{F}$, et
	l'application $\phi$ donnant l'inverse pour un opérateur $T \in
	\GSisomorphismHomo{E}{F}$ est continue.

	-- Si $T \in \GScontinueEndo{E}$ est de norme strictement inférieur à $1$,
	alors $(Id_{E} - T) \in \GSisomorphism{E}$.
\end{proposition}

\begin{proof}
	Voir cours Analyse III Crespo.
\end{proof}

Nous avons maintenant tous les outils pour démontrer
\ref{thm:spectrum_compact}, c'est-à-dire que le spectre est un compact.

\begin{proof}
	Etant en dimension finie (nous sommes sur $\complex$), il nous suffit de
	montrer que $\spectrum{T}$ est un fermé borné.

	-- $\spectrum{T}$ est borné. Supposons qu'il ne le soit pas, alors on a un
	$\lambda \in \complex$ tel que $\lambda > \norm{T}$. On a alors 
\end{proof}


Regardons quelque propriétés sur les spectres. Prenons un opérateur $T \in
\GSisomorphism{E}$. Est-ce que le spectre de $T$ et de $T^{-1}$ sont liés ?

\begin{proposition}
	Soit $T \in \GSisomorphism{E}$. Alors $\spectrum{T^{-1}} =
	\GSset{\lambda^{-1} \in \complex}{\lambda \in \spectrum{T}}$
\end{proposition}

\begin{proof}
	Remarquons que comme $T \in \GSisomorphism{E}$, $0 \notin \spectrum{T}$,
	donc il n'y a pas d'ambiguité pour inverser les valeurs du spectres.
\end{proof}

Il est parfois utile de pouvoir donner le plus grand élément du spectre de $T$, qu'on
définit comme \textbf{le rayon spectrale de $T$}.

% Donner la définition et la proposition s'y rapportant

% Exercice 1.4.6
% Exercice 1.4.7 ?


% Proposition 2.1.5 pour utiliser ce qu'on a vu en cours.
