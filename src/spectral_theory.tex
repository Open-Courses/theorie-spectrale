\section{Théorie spectrale}

Le but de cette section va être de retrouver certaines propriétés en dimension
quelconque que nous avons en dimension finie pour les opérateurs linéaires, et
qui nous ont permis de construire la théorie spectrale en dimension finie.

Nous allons étudier une classe spécifique d'opérateur linéaire: les opérateurs
compacts.

Ceux-ci ont des caractéristiques intéressantes: l'injectivité et le surjectivité
ne sont plus deux choses disjointes (comme en dimension finie), leurs valeurs
propres et leur spectre coincident en dehors de l'origine (ce qui n'est pas
toujours la cas en dimension infinie), et forment un sous-espace fermé de tous
les opérateurs linéaires continus.

Nous allons principalement travailler avec des espaces de Banach pour éviter de
se soucier des opérateurs continus bijectifs qui ne sont pas bicontinus.
On étudiera les opérateurs de points fixes des opérateurs compacts. Ceux-ci ont
des caractéristiques intéressantes: l'injectivité et le surjectivité ne sont
plus deux choses disjointes (comme en dimension finie), leurs valeurs propres et
leur spectre coincident en dehors de l'origine (ce qui n'est pas toujours la cas
en dimension infinie), et forment un sous-espace fermé de tous les opérateurs.

Nous allons principalement travailler avec des espaces de Banach pour éviter de
se soucier des opérateurs bijectifs, linéaires et continus qui ne sont pas des
isomorphismes.

%Pour la première propriété, nous voulons que si $T$ est bijective, alors elle
%est un isomorphisme de $E$. Cette propriété est acceptée si $E$ est un espace de
%Banach. Nous allons donc poser comme première condition que $E$ soit un espace
%de Banach.

%Cette demande n'est pas restrictive quand on est en dimension finie. En effet,
%comme nous travaillons sur $\complex$ ou $\real$, chaque espace de dimension
%finie est de Banach.

%Un type d'opérateur, appelé \textbf{opérateur compacte}, répond à ces critères.

Rappelons qu'on peut définir un ensemble \textbf{relativement compact} par la
définition d'être contenue dans un compact, ou de manière équivalente, que son
adhérence est compacte.

Quand il n'est pas spécifie qu'on travaille exclusivement sur $\mathbb{K} =
\real$ ou $\complex$, cela signifie que c'est valable pour n'importe quel espace
vectoriel de Banach.

$E$ sera un espace de Banach, et $\Hilbert$ un espace de Hilbert.  Rappelons
qu'un espace de Hilbert est un espace de Banach en particulier, donc les
propriétés vraies dans les espaces de Banach sont vraies dans les espaces de
Hilbert.

\subsection{Opérateur compact dans les espaces de Banach}

\begin{definition}
	Soit $T \in \GScontinueHomo{E}{F}$ où $E$ et $F$ sont deux espaces
	vectoriels normés.

	Alors $T$ est dit \textbf{compacte} si $T(\GSclosedUnitBoule{E})$ est relativement compacte
	dans $F$.
\end{definition}

De manière équivalente, on peut définir les opérateurs compacts comme les
opérateurs envoyant l'adhérence de toutes parties bornées dans une partie relativement
compacte (exercice).

On note $\GScompactHomo{E}{F}$ l'ensemble des opérateurs compactes.

Montrons d'abord qu'en dimension finie, tous les opérateurs linéaires sont
compacts.

-- Structure des opérateurs compacts.

\begin{proposition}
	Soit $T \in \GShomomorphisme{E}{F}$ où $E$ et $F$ sont de dimension finie.

	Alors $T \in \GScompactHomo{E}{F}$.
\end{proposition}

\begin{proof}
	Comme $E$ est de dimension finie, la boule unitée est compacte. Par le
	théorème des bornes atteintes, $T(\GSclosedUnitBoule{E})$ est un compact.
	Donc $T$ est compact.
\end{proof}

On a donc en particulier que $\GScompactHomo{E}{F}$ est fermé en dimension
finie, et qu'en plus, c'est un espace vectoriel.
On peut montrer que cette propriété est vraie, qu'importe la dimension de
$E$ et $F$.

\begin{proposition}
	$\GScompactHomo{E}{F}$ est un sous-espace vectoriel fermé de
	$\GScontinueHomo{E}{F}$.

	C'est-à-dire que la propriété d'être compacte est 'stable par passage à la
	limite', ou encore que toute suite convergente d'opérateurs compacts a
	comme limite un opérateur compact.
\end{proposition}

\begin{proof}

	Exercice
   % -- Montrons d'abord que c'est un sous-espace vectoriel.

	%Si on prend deux opérateur compacts $S$ et $T$, ainsi que $\lambda \in
	%\mathbb{K}$, nous devons montrer que $S + T$ est compact ainsi que $\lambda S$
	%est compact.

	%On a $\adh{(S + T)(\GSclosedUnitBoule{E})} =  \adh{S(\GSclosedUnitBoule{E}) +
	%T(\GSclosedUnitBoule{E})} \subseteq \adh{S(\GSclosedUnitBoule{E})} +
	%\adh{T(\GSclosedUnitBoule{E})}$. La compacité étant stable par somme, on a bien
	%que $(S + T)$ est compact car $(S + T)(\GSclosedUnitBoule{E})$ est relativement
	%compacte.

	%On a $(\lambda S)(\GSunitBoule{E}) = S(\lambda \GSunitBoule{E})$ par
	%linéarité de $S$. Comme $S$ est compact et $\lambda \GSunitBoule{E}$ est
	%bornée, $S(\lambda \GSunitBoule{E})$ est relativement compact. Donc
	%$\lambda S$ est compact.

	%-- Montrons que $\GScompactHomo{E}{F}$ est fermé.
\end{proof}

La propriété d'être un opérateur compact est aussi stable par composition.

\begin{proposition}
	Soient $E, F, G$ des espaces de Banach.
	Soient $S \in \GScompactHomo{E}{F}$ et $T \in \GScompactHomo{F}{G}$
	Alors $T \circ S \in \GScompactHomo{E}{G}$.
\end{proposition}

\begin{proof}
	On a $S(\GSunitBoule{E})$ qui est compris dans un compact $K$ de $F$. Comme $T$ est
	continue, $T(K)$ est aussi un compact de $G$. Donc $(T \circ
	S)(\GSunitBoule{E}) \subseteq T(K)$, d'où $T \circ S$ est compact.
\end{proof}

\begin{definition}
	Soit $T \in \GScontinueHomo{E}{F}$. On dit que $T$ est \textbf{de rang fini}
	si son image $Im(T) = T(E)$ est de dimension finie.

	On note $\GSfiniteRankHomo{E}{F}$ l'ensemble des opérateurs de rang fini.
\end{definition}

% ----- Opérateurs de rang fini
\begin{proposition}
	$\GSfiniteRankHomo{E}{F}$ est un sous-espace vectoriel de
	$\GScontinueHomo{E}{F}$.
\end{proposition}

\begin{proof}
	Si on prend deux opérateur $S$ et $T$ dont l'image est de dimension finie,
	l'image de $S + T$ est de dimension inférieur ou égale à $dim(Im(S)) +
	dim(Im(T))$ qui est finie.

	Pour $\lambda \in \mathbb{K}$ et $S$ de rang fini, on a $\lambda S(E) =
	S(\lambda E) = S(E)$ qui est de dimension finie.
\end{proof}

\begin{proposition}
	On a $\GSfiniteRankHomo{E}{F} \subseteq \GScompactHomo{E}{F}$. C'est-à-dire
	que tout opérateur linéaire continue de $E$ dans $F$ avec une image de
	dimension finie est compact.
\end{proposition}

\begin{proof}

\end{proof}

\begin{proposition}
	Soit $E$ et $F$ de dimension finie.

	Alors $\GSfiniteRankHomo{E}{F} = \GScompactHomo{E}{F} =
	\GScontinueHomo{E}{F} = \GShomomorphisme{E}{F}$
\end{proposition}

\begin{proof}
	Evident d'après les propositions précédentes
\end{proof}

On remarque donc que toutes ces classes d'opérateurs ne forment qu'une seule et
même classe quand nous sommes en dimension finie. Donc, les propositions qu'on
verra concernant les opérateurs compacts restent vrais dans le cas où les
espaces de vectoriels sont de dimension finie.

% Vrai ??
%Remarquons une chose importante: nous devons travailler sur des espaces de
%Banach!

% ----- Opérateurs de Fredholm

% ----- Début des propositions sur les opérateurs compacts.

\textbf{Mardi 28 avril}

Notation : $\GScompactEndo{E}$

Remarquons que le fait que $\GScompactEndo{E}$ est un espace vectoriel, cela
implique que $K - \lambda Id_{E}$ ($\lambda \in \complex$) est compact ssi $K
\in
\GScompactEndo{E}$.

\begin{definition} [Non officielle]
	Soit $K \in \GScontinueEndo{E}$. On appelle $T = Id_{E} - K$
	\textbf{l'opérateur de point fixe de $K$}.

	Cette définition est motivée par le fait qu'on peut voir $\ker(T)$ comme
	l'ensemble des points fixes de $K$.
\end{definition}

\begin{proposition}
	\label{prop:fixe_point_closed_injective}
	Soit $K \in \GScompactEndo{E}$, et $T$ son opérateur de point fixe. Soit $F$
	fermé tel que  est injectif sur $F$.

	Alors il existe une constance $c > 0$ tel que $\forall x \in F$, $\norm{x}
	\leq c \norm{T(x)}$.

	De plus $Im(T)$ est fermé.
\end{proposition}

\begin{remarque}
	-- Ce n'est pas parce $F$ est fermé que $T(F)$ est fermé, même si $T$ est
	continue.
\end{remarque}

\begin{proof}
	Exo
\end{proof}

\begin{proposition}
	\label{prop:supplementaire_sev_ferme}
	Soit $F$ un sous-espace vectoriel fermé de $E$. Alors il existe un
	supplémentaire topologique $G$ de $F$.
\end{proposition}

\begin{proof}
	Exo
\end{proof}

\begin{proposition}
	Soit $K \in \GScompactEndo{E}$, et $T$ son opérateur de point fixe. Alors
	$\ker(T)$ est de dimension finie et $Im(T)$ est fermé.
\end{proposition}

\begin{proof}
	- Remarquons qu'en dimension finie, c'est évident (pour la fermeture, si
	nous sommes dans un espace de banach.)

	-- En particulier, toute valeur propre de $K$ est de dimension finie car $K
	- \lambda Id_{E}$ est compact.
	%Nous allons considérer $\ker(T)$ comme un espace vectoriel normé, et montrer
	%que sa boule unité est compacte. On aura donc que $\ker(T)$ est de
	%dimension finie.

	%Soit $F = \ker(T)$.
	%Prenons un $x \in \GSclosedUnitBoule{F}$. Alors $x = K(x)$ et $\norm{x} = 1$
	%par définition de $F$.
	%Donc $x \in K(E)$, et plus particulièrement, $x \in
	%K(\GSclosedUnitBoule{E})$ car $\norm{x} = 1$.
	%On en conclut que $\GSclosedUnitBoule{F} \subseteq
	%K(\GSclosedUnitBoule{E})$.

	%Par hypothèse, $K$ est un opérateur compact, donc
	%$\adh{K(\GSclosedUnitBoule{E})}$ est compact. D'où, $\GSclosedUnitBoule{F}$
	%est contenu dans un compact, est alors relativement compact, et
	%$\adh{\GSclosedUnitBoule{F}}$ est compact par définition de relativement
	%compact.

	%$\GSclosedUnitBoule{F}$ étant fermé, $\adh{\GSclosedUnitBoule{F}} =
	%\GSclosedUnitBoule{F}$. D'où $\GSclosedUnitBoule{F}$ est compact.

	%On sait alors qu'il existe un supplémentaire fermé $G$ tel que $E = \ker{T}
	%\oplus G$. D'où, $T(E) = T(G)$.

	%Par \ref{prop:fixe_point_closed_injective}, comme $T$ est injective sur $G$,
	%on a $T(E)$ qui est fermé.
	%\begin{enumerate}
		%\item

		%\item

		%\item Remarquons que l'égalité n'est qu'une conséquence de
			%\ref{ex:ker_image_ortho_adjointe}.
	%\end{enumerate}
\end{proof}

%\begin{proposition}
	%Soit $T \in \GScontinueEndo{E}$.

	%Alors $T \in \GScompactEndo{E} \equiv \adjointe{T} \in
	%\GScompactEndo{\GSdual{E}}$.
%\end{proposition}

%\begin{proof}

%\end{proof}

\begin{exercice}
	Soit $F$ un sous-espace vectoriel fermé de $E$, et $x \in F$, $\alpha \in
	\mathbb{K}$, $y \notin F$. Alors:

	% TODO !!!!
	-- $\exists x_{0} \in F$ tel que $d(y, F) = d(x_{0}, F)$

	-- $d(\alpha y, F) = \alpha d(y, F)$

	-- $d(y - x, F) = d(y, F)$
\end{exercice}

\begin{proposition}
	Soit $E$ un espace vectoriel. Soit $F$ et $G$ deux sous-espaces vectoriels
	de $E$ tel que $F \subsetneq G$

	Supposons $F$ est fermé.

	Alors $\forall \epsilon \in ]0, 1[$, $\exists y \in G$, $\norm{y} = 1$ et
	$d(y, F) > 1 - \epsilon$

	En d'autres termes, on peut se rapprocher autant qu'on le souhaite de $F$
	tout en restant \textbf{strictement} dans $G$. (La norme n'étant pas
	importante car il suffit de normaliser).
\end{proposition}

\begin{proof}
	Soient $\epsilon \in ]0, 1[$, et $y_{0} \in G$ tel que $y_{0} \notin F$.

	On a $d(y_{0}, F) > 0$, et en particulier, on a donc $d(y_{0}, F) <
	\frac{d(y_{0},F)}{1 - \epsilon}$ car $\epsilon < 1$.

	Comme $F$ est fermé, on a un $x_{0} \in F$ tel que $d(y_{0}, x_{0}) =
	d(y_{0}, F)$ (et $d(y_{0}, x_{0}) = \norm{y_{0} - x_{0}}$).

	Prenons $y = \frac{(x_{0} - y_{0})}{\norm{x_{0} - y_{0}}}$. On a:

	\begin{align}
		d(y, F) &= d(\frac{(x_{0} - y_{0})}{\norm{x_{0} - y_{0}}}, F) \\
		&= \norm{x_{0} - y_{0}}^{-1} d(x_{0} - y_{0}, F) \\
		&= \norm{x_{0} - y_{0}}^{-1} d(y_{0}, F) \\
		&> 1 - \epsilon
	\end{align}
\end{proof}

\begin{proposition}
	Soit $K \in \GScontinueEndo{E}$ et $T$ son opérateur de point fixe.

	Alors il n'existe de pas de chaine infinie strictement croissante de sev
	fermés $\GSsequence{F}{n}{\naturel}$ tel que $T(F_{n + 1}) \subseteq
	F_{n}$.

	De même, il n'existe pas de chaine infinie strictement décroissante de
	sev fermés $\GSsequence{F}{n}{\naturel}$ tel que $T(F_{n}) \subseteq F_{n +
	1}$.
\end{proposition}

\begin{proof}
	Remarquons que le proposition est évidente en dimension finie.
\end{proof}

\begin{corollary}
	\label{cor:ker_im_stationnaire}
	Soit $K \in \GScompactEndo{E}$ et $T$ son opérateur de point fixe.

	Alors $\exists n_{0} \in \naturel$ tel que $\ker(T^{n}) = \ker(T^{n_{0}})$
	pour tout $n > n_{0}$.

	De même, on a un même $m_{0} \in \naturel$ tel que pour tout $m > m_{0}$,
	$Im(T^{m}) = Im(T^{m_{0}})$.

	En d'autres termes, la suite des noyaux et des images sont stationnaires.
\end{corollary}

\begin{proof}

\end{proof}

Nous sommes maintenant prêt à prouver l'une de nos motivations.

\begin{proposition}
	\label{prop:equiv_inj_surj_bij}
	Soit $K \in \GScompactEndo{E}$, et $T$ son opérateur de point fixe.

	Alors on a:

	$T$ injectif $\equiv$ $T$ surjectif $\equiv$ $T \in \GSisomorphism{E}$.
\end{proposition}

\begin{proof}
	Par l'absurde, en utilisant \ref{cor:ker_im_stationnaire}
\end{proof}

\subsection{Opérateur compact dans un espace de Hilbert}

Prop 2.1.7 !!!!

\subsection{Spectre d'un opérateur compact dans un espace de Banach}

Rappelons que la boule unité fermée est compacte uniquement en dimension finie.

Comme mentionné le titre de la section, on étudiera le cas des opérateurs
compacts uniquement sur le corps des réels ou le corps des complexes. La
distinction sera faite en spécifiant $\real$ ou $\complex$, sinon, le symbole
$\mathbb{K}$ sera utilisé pour désigner $\real$ ou $\complex$.

\begin{definition}
	Soit $T \in \GScontinueEndo{E}$.

	On définit \textbf{le résolvant de $T$} par l'ensemble:

	\begin{equation}
		\resolvant{T} = \GSset{\lambda \in \complex}{(T - \lambda
			{Id}_{E}) \in \GSisomorphism{E}}
			\label{def:resolvant_operator}
	\end{equation}

	On pose alors pour chaque $\lambda \in \resolvant{T}$, $R_{\lambda}(T) = (T -
	\lambda Id_{E})^{-1}$.

	On définit le complémentaire de $\rho(T)$ comme \textbf{le spectre de $T$},
	et on le note $\spectrum{T}$.
	On a donc que le spectre de $T$:

	\begin{equation}
		\spectrum{T} = \GSset{\lambda \in \complex}{(T - \lambda {Id}_{E})
		\notin \GSisomorphism{E}}
		\label{def:spectrum_operator}
	\end{equation}
\end{definition}

-- Donner la définition en générale.

Le spectre est donc un sous-ensemble de $\complex$.

L'exemple du shift nous montre que les valeurs propres ne couvrent pas tout le
spectre.

\begin{exercice}
	Calculer les valeurs propres de l'opérateur de shift à droite, et son
	spectre (disque unité).
\end{exercice}

Cependant, comme nous l'avons vu, le spectre et les valeurs propres forment le
même ensemble en dimension finie (théorème du rang).

\begin{proposition}
	Soit $T \in \GScontinueEndo{E}$.
	Soit $\lambda \in \complex$ valeur propre de $T$.

	Alors $\lambda \in \spectrum{T}$.
\end{proposition}

\begin{proof}
	Si $\lambda$ est valeur propre, alors $T - \lambda Id_{E}$ n'est pas
	inversible car il existe $v \in E$ tel que $v \neq 0_{E}$ et $T(v) - \lambda
	v = 0$. Donc $v \in \ker{T - \lambda Id_{E}}$, et donc $T - \lambda Id_{E}$
	n'est pas injective.
\end{proof}

Nous avons donc que le spectre de $T$ contient toutes les valeurs propres de
$T$.

Les isomorphismes de $E$ sont étroitement liées à la norme définie sur $E$. On
peut donc penser que le spectre diffère d'une norme à l'autre, ou du moins
qu'elle a un lien avec la topologie de la norme définie sur $E$.

Le spectre est en effet liée à la topologie, et possède une propriété commune à
chaque norme.

\begin{theorem}
	\label{thm:spectrum_compact}
	Soit $T \in \GScontinueEndo{E}$.
	Le spectre de $T$ est un compact de $\complex$.
\end{theorem}

Avant de montrer cette théorème, nous allons démontrer quelques lemmes, et
donner quelques rappels.

\begin{lemma}
	$T - \lambda Id_{E}$ et $T - \lambda_{0} Id_{E}$ commutent.
\end{lemma}

\begin{proof}

\end{proof}

\begin{proposition} [Identité de la résolvante]
	\label{prop:resolvante_identity}
	Soit $T \in \GScompactEndo{E}$.
	On a pour tout $\lambda, \mu \in \resolvant{T}$:

	\begin{equation}
		R_{\lambda}(T) - R_{\mu}(T) = (\lambda - \mu) (R_{\lambda}(T) \circ
		R_{\mu}(T))
	\end{equation}
\end{proposition}

\begin{proof}

\end{proof}

\begin{proposition}
	-- \GSisomorphismHomo{E}{F} est un ouvert de $\GScontinueHomo{E}{F}$, et
	l'application $\phi$ donnant l'inverse pour un opérateur $T \in
	\GSisomorphismHomo{E}{F}$ est continue.

	-- Si $T \in \GScontinueEndo{E}$ est de norme strictement inférieur à $1$,
	alors $(Id_{E} - T) \in \GSisomorphism{E}$.
\end{proposition}

\begin{proof}
	Voir cours Analyse III Crespo.
\end{proof}

Nous avons maintenant tous les outils pour démontrer
\ref{thm:spectrum_compact}, c'est-à-dire que le spectre est un compact.

\begin{proof}
	Etant en dimension finie (nous sommes sur $\complex$), il nous suffit de
	montrer que $\spectrum{T}$ est un fermé borné.

	-- $\spectrum{T}$ est borné. Supposons qu'il ne le soit pas, alors on a un
	$\lambda \in \complex$ tel que $\lambda > \norm{T}$. On a alors
\end{proof}


Regardons quelque propriétés sur les spectres. Prenons un opérateur $T \in
\GSisomorphism{E}$. Est-ce que le spectre de $T$ et de $T^{-1}$ sont liés ?

\begin{proposition}
	Soit $T \in \GSisomorphism{E}$. Alors $\spectrum{T^{-1}} =
	\GSset{\lambda^{-1} \in \complex}{\lambda \in \spectrum{T}}$
\end{proposition}

\begin{proof}
	Remarquons que comme $T \in \GSisomorphism{E}$, $0 \notin \spectrum{T}$,
	donc il n'y a pas d'ambiguité pour inverser les valeurs du spectres.
\end{proof}

Il est parfois utile de pouvoir donner le plus grand élément du spectre de $T$, qu'on
définit comme \textbf{le rayon spectrale de $T$}.

\begin{proposition}
	2.2.13
\end{proposition}
% Donner la définition et la proposition s'y rapportant

% Exercice 1.4.6
% Exercice 1.4.7 ?


% Proposition 2.1.5 pour utiliser ce qu'on a vu en cours.
\subsection{Spectre d'un opérateur compact dans un espace de Hilbert}
